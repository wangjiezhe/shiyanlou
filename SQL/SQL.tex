\documentclass[a4paper, twoside]{article}
\usepackage{ctex}
%\usepackage{indentfirst}
\usepackage[top=1in,bottom=1in,left=1.25in,right=1.25in]{geometry}
\usepackage{longtable}
\usepackage{booktabs}
\usepackage{hyperref}
\usepackage{titlesec}

\usepackage{fancyhdr}
\fancyhead{}
\fancyfoot{}
\fancyhead[RO,LE]{SQL~语言简介}
\fancyfoot[C]{\thepage}

\newcommand{\tightlist}{%
  \setlength{\itemsep}{0pt}\setlength{\parskip}{0pt}}

\renewcommand{\quote}{%
  > > }

\renewcommand{\contentsname}{\Large{目~录}}

\begin{document}

%\setlength{\parindent}{2em}
%\setlength{\baselineskip}{1.8em}
%\setlength{\parskip}{1ex}
\titleformat{\section}{\centering\Large\bfseries}{\thesection\,}{1em}{}

\title{\Huge{SQL 语言简介}}
\author{王介哲}

\maketitle
\pagenumbering{roman}
\tableofcontents
\clearpage

\pagestyle{fancy}

\section{查看}
\pagenumbering{arabic}

\subsection{查看数据库}

\begin{verbatim}
SHOW DATABASES;
\end{verbatim}

\subsection{连接数据库}

\begin{verbatim}
USE <database>
\end{verbatim}

\subsection{查看表}

\begin{verbatim}
SHOW TABLES;
\end{verbatim}

\section{新数据库}

\subsection{新建数据库}

\begin{verbatim}
CREATE DATABASE <database>;
\end{verbatim}

\subsection{新建数据表}

\begin{verbatim}
CREATE TABLE <table>
(
<colomn 1> <type>(<length>),
<colomn 2> <type>(<length>),
<colomn 3> <type>(<length>)
);
\end{verbatim}

\subsection{常见数据类型}

\begin{itemize}
\tightlist
\item
  数字: \texttt{INT}, \texttt{FLOAT}, \texttt{DOUBLE},
\item
  枚举: \texttt{ENUM}, \texttt{SET},
\item
  时间: \texttt{DATE}, \texttt{TIME}, \texttt{YEAR},
\item
  字符串: \texttt{CHAR}, \texttt{VARCHAR}, \texttt{TEXT}
\end{itemize}

\subsection{查看所有数据}

\begin{verbatim}
SELECT * FROM <table>;
\end{verbatim}

\subsection{插入数据}

\begin{verbatim}
INSERT INTO <table> VALUES(<all values seperated by comma>);

INSERT INTO <table>(<col 1>, <col 2>, <col 3>) VALUES(<value 1>, <value 2>, <value3>);
\end{verbatim}

\section{约束}

\begin{longtable}[c]{@{}llllll@{}}
\toprule
约束类型: & 主键 & 默认值 & 唯一 & 外键 & 非空\tabularnewline
\midrule
\endhead
关键字: & PRIMARY KEY & DEFAULT & UNIQUE & FOREIGN KEY & NOT
NULL\tabularnewline
\bottomrule
\end{longtable}

\begin{quote}
例子参见 \href{../cs9/SQL3/MySQL-03-01.sql}{MySQL-03-01.sql}
\end{quote}

\section{SELECT}

\subsection{基本格式}

\begin{verbatim}
SELECT <colomns seperated by comma> FROM <table> WHERE <condition>;
\end{verbatim}

\subsection{条件}

\begin{itemize}
\tightlist
\item
  数学符号: \texttt{-}, \texttt{\textgreater{}}, \texttt{\textless{}},
  \texttt{\textgreater{}=}, \texttt{\textless{}=}
\item
  逻辑关系: \texttt{AND}, \texttt{OR}
\item
  在 xxx 和 yyy 之间,并包含边界:
  \texttt{BETWEEN\ \textless{}xxx\textgreater{}\ AND\ \textless{}yyy\textgreater{}}
\item
  在与不再: \texttt{IN}, \texttt{NOT\ IN}
\item
  通配符: \texttt{LIKE}

  \begin{itemize}
  \tightlist
  \item
    单个字符: \texttt{\_}
  \item
    不定个字符: \texttt{\%}
  \end{itemize}
\item
  排序:

  \begin{itemize}
  \tightlist
  \item
    升序(默认): \texttt{ORDER\ BY\ \textless{}col\textgreater{}}
  \item
    降序: \texttt{ORDER\ BY\ \textless{}col\textgreater{}\ DESC}
  \end{itemize}
\item
  分组: \texttt{GROUP\ BY\ \textless{}col\textgreater{}}
\end{itemize}

\subsection{内置函数}

\begin{longtable}[c]{@{}llllll@{}}
\toprule
函数名: & COUNT & SUM & AVG & MAX & MIN\tabularnewline
\midrule
\endhead
作用: & 计数 & 求和 & 平均值 & 最大值 & 最小值\tabularnewline
\bottomrule
\end{longtable}

\begin{quote}
可以用 \texttt{AS} 给值重命名
\end{quote}

\subsection{子查询}

\begin{verbatim}
SELECT <colomns seperated by comma> FROM <table 1>
WHERE <col 1> IN
(SELECT <col 2> FROM <table 2> WHERE <condition>);
\end{verbatim}

\begin{quote}
子查询可以扩展到3层、4层或更多层
\end{quote}

\subsection{连接查询}

\subsubsection{隐式连接}

\begin{verbatim}
SELECT <colomns seperated by comma>
FROM <table 1>, <table 2>
WHERE <table 1>.<col 1> = <table 2>.<col 2>
<other conditions>;
\end{verbatim}

\subsubsection{显式连接}

\begin{verbatim}
SELECT <colomns seperated by comma>
FROM <table 1> JOIN <table 2>
ON <table 1>.<col 1> = <table 2>.<col 2>
<other conditions>;
\end{verbatim}

\begin{quote}
三表连接的例子见 \href{../cs9/hom2.sql}{hom2.sql}
\end{quote}

\section{修改和删除}

\subsection{对数据库的修改}

\subsubsection{删除数据库}

\begin{verbatim}
DROP DATABASE <database>;
\end{verbatim}

\subsection{对表的修改}

\subsubsection{重命名表}

\begin{verbatim}
RENAME TABLE <orig> TO <dest>;

ALTER TABLE <orig> RENAME <dest>;

ALTER TABLE <orig> RENAME TO <dest>;
\end{verbatim}

\subsubsection{删除表}

\begin{verbatim}
DROP TABLE <table>;
\end{verbatim}

\subsection{对表结构(i.e.列)的修改}

\subsubsection{增加列}

\begin{verbatim}
ALTER TABLE <table> ADD COLUMN <col> <type> <constraint>;

ALTER TABLE <table> ADD <col> <type> <constraint>;

ALTER TABLE <table> ADD <col> <type> <constraint> AFTER <orig col>;

ALTER TABLE <table> ADD <col> <type> <constraint> FIRST;
\end{verbatim}

\subsubsection{删除列}

\begin{verbatim}
ALTER TABLE <table> DROP COLUMN <col>;

ALTER TABLE <table> DROP <col>;
\end{verbatim}

\subsubsection{重命名列}

\begin{verbatim}
ALTER TABLE <table> CHANGE <orig> <dest> <type> <constraint>;
\end{verbatim}

\begin{quote}
``数据类型''(<type>) 不能省略
\end{quote}

\subsubsection{改变数据类型}

\begin{verbatim}
ALTER TABLE <table> MODIFY <col> <new type>;
\end{verbatim}

\begin{quote}
修改数据类型可能导致数据丢失
\end{quote}

\subsection{对表的内容的修改}

\subsubsection{修改表中的某个值}

\begin{verbatim}
UPDATE <table> SET <col 1>=<value 1>, <col 2>=<value 2> WHERE <condition>;
\end{verbatim}

\begin{quote}
一定要有 WHERE 条件,否则会修改整个表
\end{quote}

\subsubsection{删除一行记录}

\begin{verbatim}
DELETE FROM <table> WHERE <condition>;
\end{verbatim}

\begin{quote}
一定要有 WHERE 条件,否则会删除所有数据
\end{quote}

\section{其它}

\subsection{索引}

\begin{verbatim}
ALTER TABLE <table> ADD INDEX <index> (<col>);

CREATE INDEX <index> FROM <table> (<col>);
\end{verbatim}

\subsection{视图}

\begin{verbatim}
CREATE VIEW <view>(<col a>, <col b>, <col c>)
AS SELECT <col 1>, <col 2>, <col 3> FROM <table>;
\end{verbatim}

\begin{quote}
视图是虚拟的表
\end{quote}

\subsection{导入}

\begin{verbatim}
LOAD DATA INFILE '<file path>' INTO TABLE <table>;
\end{verbatim}

\subsection{导出}

\begin{verbatim}
SELECT <colomns seperated by comma, or star(*) to select all>
INTO OUTFILE '<file path>' FROM <table>;
\end{verbatim}

\subsection{备份}

\begin{verbatim}
mysqldump -u root <database> > <backup.sql>  # 备份整个数据库

mysqldump -u root <database> <table> > <backup.sql>  # 备份整个表
\end{verbatim}

\subsection{恢复}

\begin{verbatim}
mysql -u root
    SOURCE <backup.sql>;
\end{verbatim}

或者

\begin{verbatim}
mysql -u root
    CREATE DATABASE <database>;
    exit
mysql -u root < <backup.sql>
\end{verbatim}

\end{document}
